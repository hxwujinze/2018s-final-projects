\section{静态语义}

\textbf{PCF}的语法格式定义如下:

\begin{table}[!htbp]
	\centering
	\begin{tabular}{lllll}
		Typ \(\tau\) & \(::=\) & \(\mathtt{nat}\) & \(\mathtt{nat}\) & 自然数 \\
		& & \(\mathtt{parr}(\tau_1; \tau_2)\) & \(\tau_1 \rightharpoonup \tau_2\) & \gls{partial function} \\
		Exp \(e\) & \(::=\) & \(x\) & \(x\) & 变量 \\
		& & \(\mathtt{z}\) & \(\mathtt{z}\) & 零 \\
		& & \(\mathtt{s}(e)\) & \(\mathtt{s}(e)\) & 后继 \\
		& & \(\mathtt{ifz}\{e_0; x.e_1\}(e)\) & \(\mathtt{ifz}\ e\ \{\mathtt{z} \hookrightarrow e_0 | \mathtt{s}(x) \hookrightarrow e_1\}\) & 零检验 \\
		& & \(\mathtt{lam}\{\tau\}(x.e)\) & \(\lambda\ (x:\tau)\ e\) & 抽象 \\
		& & \(\mathtt{ap}(e_1;e_2)\) & \(e_1(e_2)\) & 应用 \\
		& & \(\mathtt{fix}{\tau}(x.e)\) & \(\mathtt{fix}\ x:\tau\ \mathtt{is}\ e\) & 递归
	\end{tabular}
\end{table}

表达式\(\mathtt{fix}{\tau}(x.e)\)是\gls{general recursion},后面会有具体的讨论;
而表达式\(\mathtt{ifz}\{e_0; x.e_1\}(e)\)根据\(e\)的值是否是\(\mathtt{z}\),当不是时绑定前驱给\(x\)。

\textbf{PCF}的静态语义是由以下规则递归地定义的:
% 这里的编号是直接指定的,而不是自动产生的,请留意
\begin{gather}
	\frac{}{
		\Gamma, x:\tau \vdash x:\tau
	}
	\tag{19.1a} \\
	\frac{}{
		\Gamma \vdash \mathtt{z}:\mathtt{nat}
	}
	\tag{19.1b} \\
	\frac{
		\Gamma \vdash e: \mathtt{nat}
	}{
		\Gamma \vdash \mathtt(s)(e): \mathtt{nat}
	}
	\tag{19.1c} \\
	\frac{
		\Gamma \vdash e: \mathtt{nat} \quad 
		\Gamma \vdash e_0: \tau \quad
		\Gamma, x: \mathtt{nat} \vdash e_1: \tau
	}{
		\Gamma \vdash \mathtt{ifz}\{e_0; x.e_1\}(e): \tau
	}
	\tag{19.1d} \\
	\frac{
		\Gamma, x: \tau_1 \vdash e: \tau_2
	}{
		\Gamma \vdash \mathtt{lam}\{\tau_1\}(x.e): \mathtt{parr}(\tau_1; \tau_2)
	}
	\tag{19.1e} \\
	\frac{
		\Gamma \vdash e: \mathtt{parr}(\tau_2; \tau) \quad 
		\Gamma \vdash e_2: \tau
	}{
		\Gamma \vdash \mathtt{ap}(e_1; e_2): \tau
	}
	\tag{19.1f} \\
	\frac{
		\Gamma, x: \tau \vdash e: \tau
	}{
		\Gamma \vdash \mathtt{fix}\{\tau\}(x.e): \tau
	}
	\tag{19.1g}
	\label{equ:fix.static}
\end{gather}

规则\ref{equ:fix.static}反映了\gls{general recursion}的自我参照特性(self-referential nature)。
为了说明\(\mathtt{fix}\{\tau\}(x.e)\)具有\(\tau\)类型,我们假设它具有将该类型赋值给\(x\),其中\(x\)代表这个递归表达自身。
然后检查内部的表达式\(e\)是否在这个假设下具有类型\(\tau\)。

这些结构化的规则,特别包括替换,是可采纳的静态语义。

\paragraph{引理19.1.} 如果\(\Gamma, x: \tau \vdash e': \tau', \Gamma \vdash e: \tau\),那么\(\Gamma \vdash [e/x]e': \tau'\)
