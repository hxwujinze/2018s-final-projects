我们讨论了\gls{total computations}(total computations)所根据的\textbf{T}语言,其类型系统是保证计算终止的。
\textbf{M}语言则推广了\textbf{T}语言,以支持\gls{inductive types}(inductive types)和\gls{coinductive types}(coinductive types)。
这一章中我们会介绍\textbf{PCF}系统来讨论\gls{partial computations}(partial computations),即便它是\gls{well-typed}(良类型的),在求值时仍有可能不会终止。
% 下一句引用第20章(下一章),原pdf上有链接
这在一开始看来像是一个缺点,但我们会在第二十章中看到它所允许的比\textbf{T}语言更大的表达能力。

\textbf{PCF}的部分性来源于\gls{general recursion}概念,它允许在一些表达式之中求解等式组。承认所有这些等式组有解的代价是可能存在不终止的计算,因为会有一部分求解所需的等式还没有定义(发散)。
在\textbf{PCF}中,编程者必须确保计算能够终止,而类型系统不保证这点。这么做的好处是终止的证明不必嵌入到代码中,从而使得代码更短。

举例来说,考虑下面的等式组:
\begin{align*}
	f(0) &\triangleq 1 \\
	f(n + 1) &\triangleq (n + 1) \times f(n)
\end{align*}

凭直觉就能看出这些等式定义了阶乘函数。这些等式联立起来构成了对自然数域上的未知函数\(f\)的方程,我们所要找的解就是就是特定的函数\(f: \mathbb{N} \rightarrow \mathbb{N}\)满足上面的条件。

这样一个联立等式系统的解就是就是在一个联合函数(高阶函数)上的\gls{fixed point}(fixed point)。为了能够更好的理解,我们以另一种形式重写这些等式:

\begin{align*}
	f(n) \triangleq 
	\left\{
	\begin{array}{ll}
		1 & \mathrm{if\ } n = 0 \\
		n \times f(n') & \mathrm{if\ } n = n' + 1
	\end{array}
	\right.
\end{align*}

再重写一遍,我们要找的\(f\)要使得

\begin{align*}
	n \mapsto 
	\left\{
	\begin{array}{ll}
		1 & \mathrm{if\ } n = 0 \\
		n \times f(n') & \mathrm{if\ } n = n' + 1
	\end{array}
	\right.
\end{align*}

% 这里文中的functional和数学上的functional(传统意义上的\gls{functional})之含义是不同的,请留意是否应该使用这种翻译方式
现在,我们要定义\gls{functional}\(F\),使得\(F(f)=f'\),其中\(f'\)要满足

\begin{align*}
	n \mapsto 
	\left\{
	\begin{array}{ll}
		1 & \mathrm{if\ } n = 0 \\
		n \times f(n') & \mathrm{if\ } n = n' + 1
	\end{array}
	\right.
\end{align*}

需要注意这里\(f'\)的条件是用\gls{functional}\(F\)的参数\(f\)表示的,而不是使用\(f'\)自己表示的。
我们要找的函数\(f\)就是\(F\)的\gls{fixed point},它是一个函数\(f: \mathbb{N} \rightarrow \mathbb{N}\)满足\(f = F(f)\)。
换句话说,\(e\)被定义为\(fix(F)\),其中\(fix\)是高阶运算,作用在\gls{functional}\(F\)上并计算其\gls{fixed point}。

为什么这样的一个操作\(F\)具有\gls{fixed point}?
重点在于\textbf{PCF}中的函数是部分性的,这就意味着它们可能对一些(甚至全部)输入发散。
因此,\gls{functional}\(F\)的\gls{fixed point}是对于迭代\(F\)期待得到的一系列近似结果上的限制。
我们说一个自然数上的\gls{partial function}\(\phi\)是一个\gls{total function}\(f\)的近似,当如果\(\phi(m) = n\)的时候总有\(f(m)=n\)。
% 请注意这里用harpoonup而不是arrow,表示partial function
定义\(\perp: \mathbb{N} \rightharpoonup \mathbb{N}\)是完全没有定义的\gls{partial function},即对\(\forall n \in \mathbb{N}\),\(\perp(n)\)都是没有定义的。
这是我们所期待的上述递归式中\(f\)的解中“最差”的近似。
给定任何一个\(f\)的近似\(\phi\),我们可以用\(\phi' = F(\phi)\)进行“改进”。
于是,\gls{partial function}\(\phi'\)首先能够定义在0上,然后\(\phi\)有定义的\(\forall m > 0\)上,\(\phi'\)就能定义在相应的\(m + 1\)上。
继续这个过程,\(\phi'' = F(\phi') = F(F(\phi))\)是对\(\phi'\)的改进,以此类推可以得到更多的改进。
如果我们从\(\perp\)开始作为最初的\(f\)的近似,不断计算直到达到极限

\[
	\lim_{i \geq 0} F^{(i)}(\perp)
\]

我们至少能得到定义在每个\(m \in \mathbb{N}\)上的\(f\)的近似,因此得到的就是\(f\)自己。
反过来说,如果这个极限存在,那这就是我们想要的结果。

因为这样的构造对任何\gls{functional}\(F\)都可行,所以我们可以总结出所有这样的操作都具有不动点,从而所有的等式组系统就像上面的例子一样具有解。
这个解是由\gls{general recursion}给出的,但并不保证是一个\gls{total function}(定义在当前域中的每个元素上)。
上面的例子只是恰巧是\gls{total function},因为我们可以通过归纳来证明,但是通常情况下解会是一个\gls{partial function}并在一些输入上发散。
这是编程者的任务来保证通过\gls{general recursion}定义的函数是\gls{total function},或者至少能够控制其输入在已定义的范围内。
