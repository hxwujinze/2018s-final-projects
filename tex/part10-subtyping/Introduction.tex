子类型关系是一种类型之间的先序关系(具有自反性和传递性的关系)。
这种关系是\textit{包容原理}的一种印证:
\begin{quotation}
    \textit{如果$\tau$是$\tau{'}$的子类型,那么当类型为$\tau$的一个值被需要时,可以提供类型为$\tau{'}$的一个值。}
\end{quotation}
包容原理放宽了类型系统的约束,允许某个类型的值被当做另一个类型的值来对待。

经验表明尽管包容原理作为一个一般性的准则是很有用的,但是在某些实际情况下,
正确地应用它会是一件棘手的事情。
成功运用包容原理的关键是介入和除去规则。
通过检验子类型的所有\textit{介入形式}是否可以被父类型的所有\textit{除去形式}安全地生成,
我们可以判断一个候选的子类型关系是否合理。
子类型的定型规则只有通过了这项测试,才是有意义的。
对于任何子类型关系的类型安全定理之证明都会确保这一点。

对于子类型的定型规则的典型误解是将一个类型仅仅视为一个由介入规则生成的值的集合,
并只是检查是否每个子类型的值都可以被当做父类型的值。
这种做法背后的思维是把子类型的定型当作是一般数学意义上的子集关系的近亲。
然而,正如我们接下来会看到的,这种思维会导致严重的错误,
因为它没有考虑到可应用到父类型上的除去形式。
仅仅考虑介入规则是不够的,子类型的定型是一种\textit{行为}而不仅仅是某种\textit{包含}关系。